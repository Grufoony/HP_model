Bioinformatics is conceptualizing biology in terms of molecules and applying ``informatics techniques'' to understand and organize the information associated with these molecules, on a large scale \cite{bioinfo}.
The aims of bioinformatics are multiple.
Firstly, well-organized data allow researches to access existing information and to submit new entries as they are produced.
Secondly, it develops tool and resources useful for the analysis of these data.
The development of these tools requires a lot of knowledge in biology, informatic and also physics and mathematics.

In particular, this project will focus on the protein folding problem.
How protein folds is a vital process that can be useful for a lot of purposes, from bioengineering to the medical applications.
Understanding how proteins fold can help us to cure diseases like Alzheimer, in which proteins start to not fold correctly, treat virus infections and design more efficient drugs \cite{PERM}. 

The aim of this project is to report the main results in the developing of the HP model for protein folding.
In order to do so, we'll first illustrate what the HP model is.
Next, we will analyze two main approaches to this problem.
The first one is a \emph{brute force} approach, that allows us to understand the properties of a protein by computing all its possible configurations.
However, a brute force approach is typically the last thing one would do.
An alternative will be the Reinforcement Learning method that using the so-called \emph{Deep Neural Networks} (DNNs) it's able to find the minimum of the free energy in a tolerable amount of time.