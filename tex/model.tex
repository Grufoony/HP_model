The Hydrophobic-Polar model is a very simple model used to analyze the protein folding phenomenon \cite{PERM}.
The goal is, like in the complex systems field, to simplify as much as possible the problem maintaining consistency with experimental data.
In this model, the 20 types of amino acids present in nature are divided into 2 classes: the hydrophobic ones, labelled by \emph{H}, and the hydrophilic (or polar) ones, labelled by \emph{P}.
So, taking a sequence of amino acids of length $k$, we need to convert it in an \emph{HP} sequence, i.e.
\begin{equation*}
    s = \left(s_1, \ldots, s_k\right) \ \ \ \ s_i \in \left\{H, P\right\} \forall i \in \left[1,\ldots,k\right]
\end{equation*}
Once obtained the sequence we need to define in which space $\mathbb{G}$ we want to fold it.
To simplify the problem, we may think the protein folds on a square cubic lattice $\mathbb{G} = \mathbb{Z}^2$ or a three-dimensional cubic lattice $\mathbb{G} = \mathbb{Z}^3$.
We can now define a \emph{fold} as an injective mapping $f : \left[1,\ldots,k\right] \to \mathbb{G}$ and denote with $\mathcal{Z}_k$ the set of all folds with length $k$.
It is clear that at each point $f(i)$ is assigned an amino acid $s_i$.
In order to define a way to express the energy of a fold we need to choose a function differentiates between adjacent and connected amino acids
\begin{equation*}
    \delta_{f(i),f(j)} =
    \begin{cases}
        1 \ , \ i \pm 1 \neq j\\
        0 \ , \ \text{otherwise}
    \end{cases}
\end{equation*}
Now, we've seen experimentally that folded proteins tends to form hydrophobic nuclei, so we set to $-1$ the energy value of a single $H$-$H$ bond, and more in general
\begin{equation*}
    \epsilon_{i,j} =
    \begin{cases}
        -1 \ , \ s_i = s_j = H\\
        0 \ , \ \text{otherwise}
    \end{cases}
\end{equation*}
So the total energy of the fold $f$ is
\begin{equation*}
    \mathcal{E}(f) = \sum_{0 \leq i \leq j \leq k} \epsilon_{i,j}\delta_{f(i),f(j)}
\end{equation*}
The goal of the \emph{HP} model is then, like most physical problems, to minimize the energy
\begin{equation*}
    \mathcal{E}_\text{stable} = \min_{f \in \mathcal{Z}_k} \mathcal{E}(f)
\end{equation*}
