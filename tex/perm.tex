It has been shown that the \emph{HP} model is a NP hard problem, so it's very difficult to fold efficiently longer protein sequences.
We will now discuss the Rosenbluth sampling method \cite{PERM}, which involves drawing successive steps of a random walk only from among acceptable points, which are points previously not visited.

\subsection{Self-Avoiding Walks}
In a random walk process on $\mathbb{G} = \mathbb{Z}^3$, each iteration consist in a step among one out of six equally likely directions.
An analogue result is valid for $\mathbb{G} = \mathbb{Z}^2$ where we can choose one out of four equally likely directions.
Our goal is to simulate the folding of a protein of length $k$ in the space $\mathbb{G}$ by doing a random walk of $k$ steps.
The first thing we can notice is that this cannot be a regular random walk process because the fold is not allowed to cross itself or back up on itself at any iteration.
The direct consequence is that each iteration of the random walk will have some constraints, e.g. all steps after the first one will have a forbidden direction (because they cannot back up).
At this point, we can define as \emph{Self-Avoiding Walk} (SAW) a lattice path that does not visit the same point more than once.
It is known that SAWs are fractals and their number on a given lattice will increase exponentially with the length \cite{madras1988pivot}.

\subsection{Estimate the number of folds}
For simplicity, let's assume that our SAW starts form the origin of our lattice.
At any iteration $m$ we will have only $w_m$ possible moves, where $w$ represents a sort of weight of our chain.
Let's call $W$ the weight of the total fold: a fold of length $k$ will have a weight
\begin{equation*}
    W_k = \prod_{m=1}^k w_m
\end{equation*}
It is not excluded that at a certain step $m < k$ we may have no further possibilities for continuation: we then say that a \emph{non-extendable} fold of length $m$ has been formed.
Let's denote with $\mathcal{K}$ the set of all folds with length $m \leq k$.
Clearly $\mathcal{Z}_k \subset \mathcal{K}$.
We notice that the probability of picking a random fold $f_i \in \mathcal{K}$ of length $k$ is
\begin{equation*}
    \mathbb{P}\left(f_i \in \mathcal{Z}_k\right) = \prod_{j=1}^k \frac{1}{w_j} = \frac{1}{W_k}
\end{equation*}
Let's now denote with $n_y$ the number of folds $f_i$ for which $W\left(f_i\right) = y$ and the set of these elements $\mathcal{W}_y = \left\{f \in \mathcal{K} \ | \ W\left(f_i\right) = y\right\}$.
Then, for the law of large numbers
\begin{equation*}
    \frac{n_s}{n} \approx \mathbb{P}\left(\mathcal{W}_y\right)
\end{equation*}
In the same way we can notice that
\begin{equation*}
    \left\langle W \right\rangle = \frac{1}{n} \sum_{i=1}^n W\left(f_i\right) \approx \sum_f \mathbb{P}\left(f\right)W\left(f\right) = \sum_{f \in \mathcal{Z}_k} 1 = \left\lvert \mathcal{Z}_k \right\rvert
\end{equation*}
So we've found an estimator for the number of folds
\begin{equation}
    \hat{\mathcal{Z}_k} = \left\langle W \right\rangle \approx \left\lvert \mathcal{Z}_k \right\rvert
\end{equation}